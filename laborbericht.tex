\documentclass[a4paper,12pt,fleqn,oneside]{article}
\usepackage{graphicx}
\usepackage{etex}
\usepackage[latin1]{inputenc}
\usepackage[ngerman]{babel}
\usepackage{ae,aecompl}
\usepackage[T1]{fontenc}
\usepackage{ngerman}
\usepackage{float}
\usepackage{ulem}
\usepackage{amssymb}
\usepackage[locale=DE, per-mode=fraction, quotient-mode=fraction, group-minimum-digits=6,]{siunitx}
\usepackage{tabularx}
\usepackage{bm}
\usepackage{booktabs}
\usepackage{color}
\usepackage{pictex}
\usepackage[left=2.5cm,right=2.5cm,top=2cm,bottom=2cm,includeheadfoot]{geometry}
\usepackage[section]{placeins}
\usepackage{xspace}
\usepackage{multirow}
\usepackage{lastpage}
\usepackage{fancyhdr}
\usepackage{graphicx}
\usepackage{esvect}
\usepackage{pgfplots}
\usepackage[ngerman]{babel}
\usepackage {graphics}
\usepackage {graphicx}
\usepackage{tikz}
\usepackage{amsmath}
\usepackage{autonum}


\setlength{\headheight}{15pt}
\pagestyle{fancy}
\fancyfoot[C]{Seite \thepage{} von \pageref{LastPage}}
\linespread{1.5}
\author{Dominik Eisele}
\title{Laborbericht}
\date{\today}


%\begingroup\makeatletter\ifx\SetFigFontNFSS\undefined%
%\gdef\SetFigFontNFSS#1#2#3#4#5{%
%  \reset@font\fontsize{#1}{#2pt}%
%  \fontfamily{#3}\fontseries{#4}\fontshape{#5}%
%  \selectfont}%
%\fi\endgroup%


% Zus�tzliche Spalten mit variabler Breite fur tabularx
%\newcolumntype{L}{>{\raggedright\arraybackslash}X} % linksb�ndig
%\newcolumntype{C}{>{\centering\arraybackslash}X} % zentriert
%\newcolumntype{R}{>{\raggedleft\arraybackslash}X} % rechtsb�ndig


\newcolumntype{L}[1]{>{\raggedright\arraybackslash}p{#1}} % linksb�ndig mit Breitenangabe
\newcolumntype{C}[1]{>{\centering\arraybackslash}p{#1}} % zentriert mit Breitenangabe
\newcolumntype{R}[1]{>{\raggedleft\arraybackslash}p{#1}} % rechtsb�ndig mit Breitenangabe


\setlength{\tabcolsep}{0pt}
\renewcommand{\arraystretch}{1}


%\renewcommand*\contentsname{Gliederung}


\let\oldsqrt\sqrt
\def\sqrt{\mathpalette\DHLhksqrt}
\def\DHLhksqrt#1#2{\setbox0=\hbox{$#1\oldsqrt{#2\,}$}\dimen0=\ht0
\advance\dimen0-0.3\ht0
%0.3 ist das Ma� f�r die Hakenl�nge, relativ zum Inhalt der Wurzel
\setbox2=\hbox{\vrule height\ht0 depth -\dimen0}%
{\box0\lower0.4pt\box2}}




\newcommand{\Vo}[1]{{$\SI{#1}{\volt}$}}
\newcommand{\Am}[1]{{$\SI{#1}{\ampere}$}}
\newcommand{\mAm}[1]{{$\SI{#1}{\milli\ampere}$}}
\newcommand{\Se}[1]{{$\SI{#1}{\second}$}}




\begin{document}

\begin{titlepage}
	\begin{flushleft}
		\vspace*{2\baselineskip}
		{\fontsize{16}{19.2}\selectfont Laborbericht Physik TGE12/2 A}\\[4\baselineskip]
		\begin{tabularx}{\textwidth}{rp{5px}X}
			{\fontsize{16}{19.2}\selectfont Titel:}&&{\fontsize{16}{19.2}\selectfont Entladung eines Kondensators}
		\end{tabularx}
		\\[5\baselineskip]
		\setlength{\tabcolsep}{0pt}
		\renewcommand{\arraystretch}{1,25}
		\begin{tabular}{lp{5px}l}
			{\fontsize{14}{16.8}\selectfont Bearbeiter:}&&{\fontsize{14}{16.8}\selectfont Dominik Eisele{}}\\
			{\fontsize{14}{16.8}\selectfont Mitarbeiter:}&&{\fontsize{14}{16.8}\selectfont }\\
			{\fontsize{14}{16.8}\selectfont Datum Versuchsdurchf�hrung:}&&{\fontsize{14}{16.8}\selectfont 29.02.2016}\\
			{\fontsize{14}{16.8}\selectfont Datum Abgabe:}&&{\fontsize{14}{16.8}\selectfont 23.03.2016}
		\end{tabular}
		\\[2\baselineskip]
		{\fontsize{14}{16.8}\selectfont Ich erkl�re an Eides statt, den vorliegenden Laborbericht selbst angefertigt zu haben. Alle fremden Quellen wurden in diesem Laborbericht benannt.}
		\\[2\baselineskip]
		{\fontsize{14}{16.8}\selectfont Aichwald, 16. M�rz 2016 $  $ Dominik Eisele}
	\end{flushleft}
\end{titlepage}

\setlength{\tabcolsep}{7pt}
\renewcommand{\arraystretch}{1,7}

\newpage
\tableofcontents
\newpage


\section{Einf�hrung}
	Bei dem Versuch zur Entladung eines Kondensators ergibt sich die Entladekurve eines Kondensators �ber einen Widerstand.
	Aus dieser Entladekurve lassen sich nun die Ladung und die Kapazit�t berechnen. \\
	Zus�tzlich wurde f�r den Widerstand, �ber den der Kondensator entladen wird, das Multimetereingesetzt. So kann man den
	Innenwiderstand des Multimeters berechnen.

\subsection{Ziele}
	\begin{itemize}
		\item Aufnahme der Entladekurve eines Kondensators
		\item Bestimmung der Ladung eines Kondensators (aus der Entladekurve)
		\item Bestimmung der Kapazit�t eines Kondensators
		\item Bestimmung des Innenwiderstands eines Multimeters
	\end{itemize}

\subsection{Formeln}
	Strom/Ladung
	\[I=\frac{\Delta Q}{\Delta t} \text{ bzw. }I=\frac{dQ}{dt} \text{; } [I]=\si{\ampere}\]
	Kapazit�t
	\[C=\frac{Q}{U} \text{; } [C]=\si{\farad}\]
	Entladung eines Kondensators �ber einen Widerstand
	\[Q(t)=Q_{0} \cdot e^{-\frac{t}{RC}}\]
	\[U(t)=U_{0} \cdot e^{-\frac{t}{RC}}\]

\newpage
\section{Material und Methoden}

\subsection{Material}
	F�r den Versuch verwendete Materialien:
	\begin{itemize}
		\item 1 $\times$ Netzteil
		\item 1 $\times$ Digitalmultimeter
		\item 1 $\times$ Kondensator
		\item 1 $\times$ Widerstand \SI{47}{\kilo\ohm}
		\item 1 $\times$ Steckbrett
		\item Leitungen
	\end{itemize}

\subsection{Aufbau}
	Die Schaltung, wie sie in Abbildung \ref{fig:skizze_schaltplan} zu sehen ist, wurde auf dem Steckbrett gesteckt. \\
	In der zweiten Messreihe wurde der Widerstand entfernt, sodass der Kondensator nur �ber den Innenwiderstand des
	Multimeters entladen wurde.
	\begin{figure}[H]
		\centering
		\scalebox{1.5}{%Title: /tmp/xfig-fig037018
%%Created by: fig2dev Version 3.2 Patchlevel 5e
%%CreationDate: Thu Mar 31 20:55:23 2016
%%User: dominik@deepthought.private-site.de (Dominik Eisele)
\font\thinlinefont=cmr5
%
\begingroup\makeatletter\ifx\SetFigFont\undefined%
\gdef\SetFigFont#1#2#3#4#5{%
  \reset@font\fontsize{#1}{#2pt}%
  \fontfamily{#3}\fontseries{#4}\fontshape{#5}%
  \selectfont}%
\fi\endgroup%
\mbox{\beginpicture
\setcoordinatesystem units <1.04987cm,1.04987cm>
\unitlength=1.04987cm
\linethickness=1pt
\setplotsymbol ({\makebox(0,0)[l]{\tencirc\symbol{'160}}})
\setshadesymbol ({\thinlinefont .})
\setlinear
%
% Fig POLYLINE object
%
\linethickness= 0.500pt
\setplotsymbol ({\thinlinefont .})
{\color[rgb]{0,0,0}\putrule from  1.611 22.672 to  1.611 22.003
}%
%
% Fig POLYLINE object
%
\linethickness= 0.500pt
\setplotsymbol ({\thinlinefont .})
{\color[rgb]{0,0,0}\putrule from  1.611 23.438 to  1.611 23.973
}%
%
% Fig POLYLINE object
%
\linethickness= 0.500pt
\setplotsymbol ({\thinlinefont .})
{\color[rgb]{0,0,0}\putrule from  0.212 22.441 to  0.212 22.003
\putrule from  0.212 22.003 to  2.974 22.003
\putrule from  2.974 22.003 to  2.974 22.441
}%
%
% Fig POLYLINE object
%
\linethickness= 0.500pt
\setplotsymbol ({\thinlinefont .})
{\color[rgb]{0,0,0}\putrule from  0.212 23.535 to  0.212 23.973
\putrule from  0.212 23.973 to  2.974 23.973
\putrule from  2.974 23.973 to  2.974 23.535
}%
%
% Fig TEXT object
%
\put{R} [lB] at  3.260 22.9
%
% Fig TEXT object
%
\put{C} [lB] at  2.007 22.9
%
% Fig TEXT object
%
\put{V} [lB] at  0.667 22.9
%
% Fig POLYLINE object
%
\linethickness= 0.500pt
\setplotsymbol ({\thinlinefont .})
{\color[rgb]{0,0,0}\putrule from  2.974 23.535 to  2.974 23.317
}%
%
% Fig POLYLINE object
%
\linethickness= 0.500pt
\setplotsymbol ({\thinlinefont .})
{\color[rgb]{0,0,0}\putrule from  1.611 23.110 to  1.611 23.438
}%
%
% Fig ELLIPSE
%
\linethickness= 0.500pt
\setplotsymbol ({\thinlinefont .})
{\color[rgb]{0,0,0}\ellipticalarc axes ratio  0.322:0.328  360 degrees 
	from  0.533 22.989 center at  0.212 22.989
}%
%
% Fig POLYLINE object
%
\linethickness= 0.500pt
\setplotsymbol ({\thinlinefont .})
{\color[rgb]{0,0,0}\putrule from  1.399 23.110 to  1.831 23.110
}%
%
% Fig POLYLINE object
%
\linethickness= 0.500pt
\setplotsymbol ({\thinlinefont .})
{\color[rgb]{0,0,0}\putrule from  1.399 23.000 to  1.831 23.000
}%
%
% Fig POLYLINE object
%
\linethickness= 0.500pt
\setplotsymbol ({\thinlinefont .})
{\color[rgb]{0,0,0}\putrule from  1.611 23.000 to  1.611 22.672
}%
%
% Fig POLYLINE object
%
\linethickness= 0.500pt
\setplotsymbol ({\thinlinefont .})
{\color[rgb]{0,0,0}\putrule from  0.212 22.659 to  0.212 22.441
}%
%
% Fig POLYLINE object
%
\linethickness= 0.500pt
\setplotsymbol ({\thinlinefont .})
{\color[rgb]{0,0,0}\plot  0.212 22.989  0.114 22.879 /
}%
%
% Fig POLYLINE object
%
\linethickness= 0.500pt
\setplotsymbol ({\thinlinefont .})
{\color[rgb]{0,0,0}\putrule from  0.212 23.317 to  0.212 23.535
}%
%
% Fig POLYLINE object
%
\linethickness= 0.500pt
\setplotsymbol ({\thinlinefont .})
{\color[rgb]{0,0,0}\plot  0.212 22.989  0.432 23.207 /
%
% arrow head
%
\plot  0.315 23.015  0.432 23.207  0.238 23.092 /
%
}%
%
% Fig POLYLINE object
%
\linethickness= 0.500pt
\setplotsymbol ({\thinlinefont .})
{\color[rgb]{0,0,0}\putrule from  2.974 22.659 to  2.974 22.441
}%
%
% Fig POLYLINE object
%
\linethickness= 0.500pt
\setplotsymbol ({\thinlinefont .})
{\color[rgb]{0,0,0}\color[rgb]{0,0,0}\putrectangle corners at  2.866 23.317 and  3.084 22.659
}%
\linethickness=0pt
\putrectangle corners at -0.127 23.999 and  3.291 21.977
\endpicture}
}
		\caption{Skizze des Schaltplans}
		\label{fig:skizze_schaltplan}
	\end{figure}


\newpage
\subsection{Durchf�hrung}
	Nachdem die Schaltung wie beschrieben aufgebaut worden war, wurde der Kondensator f�r \SI{30}{\second} mit \SI{5}{\volt}aufgeladen.
	Als der Kondensator vollst�ndig geladen war, wurde das Netzteil ausgesteckt, sodass sich der Kondensator entladen konnte.
	Die Spannung die am Kondensator anlag wurde, w�hrend des Entladevorgangs, in \SI{10}{\second} Intervallen gemessen. Die
	Messungen wurden abgebrochen, nachdem \SI{3}{\percent} der Ausgangsspannung  anlagen, bzw. \SI{5}{\minute} vergangen waren.\\
	Diese Messung wurde mit \SI{10}{\volt} und mit \SI{15}{\volt} wiederholt. Zus�tzlich wurde eine Messreihe durchgef�hrt, bei der der
	Widerstand entfernt wurde. Damit hat man die M�glichkeit den Innenwiderstand des Multimeters zu berechnen. Diese Messreihe wurde
	bei eine Audgangsladung von \SI{10}{\volt} durchgef�hrt.
	Diese Messung wurde mit \SI{10}{\volt} und mit \SI{15}{\volt} wiederholt.\\
	Zus�tzlich wurde eine Messreihe durchgef�hrt, bei der der Widerstand entfernt wurde. Damit hat man die M�glichkeit den
	Innenwiderstand des Multimeters zu berechnen. Diese Messreihe wurde bei eine Audgangsladung von \SI{10}{\volt} durchgef�hrt.



\newpage
\section{Messwerte}
	In Tabelle \ref{tab:messwerte1} sind die Messwerte f�r die ersten drei Messreihen zu sehen. Hierbei handelt es sich um die
	Kondensatorentladungen �ber den Widerstand. In Tabelle \ref{tab:messwerte2} sind die Messwerte f�r die Entladung �ber den
	Innenwiderstand des Multimeters zu sehen.
 



	\begin{table}
		 \centering
 \begin{tabular}{llllllllll}
         \cline{1-4} \cline{7-10}
         $t$            & \Vo{5   }       & \Vo{10  }      & \Vo{15   }        &  &  & $t$         & \Vo{5    }      & \Vo{10   }    & \Vo{15   }   \\ \cline{1-4} \cline{7-10}
         \Se{0  }       & \Vo{5,0 }       & \Vo{10,0}      & \Vo{15,0 }        &  &  & \Se{160}    & \Vo{0,2  }      & \Vo{0,38 }    & \Vo{0,58 }   \\
         \Se{10 }       & \Vo{4,0 }       & \Vo{8,17}      & \Vo{12,16}        &  &  & \Se{170}    & \Vo{0,2  }      & \Vo{0,31 }    & \Vo{0,48 }   \\
         \Se{20 }       & \Vo{3,2 }       & \Vo{6,54}      & \Vo{9,85 }        &  &  & \Se{180}    & \Vo{0,14 }      & \Vo{0,26 }    & \Vo{0,40 }   \\
         \Se{30 }       & \Vo{2,7 }       & \Vo{5,30}      & \Vo{7,99 }        &  &  & \Se{190}    & \Vo{0,12 }      & \Vo{0,21 }    & \Vo{0,33 }   \\
         \Se{40 }       & \Vo{2,15}       & \Vo{4,34}      & \Vo{6,54 }        &  &  & \Se{200}    & \Vo{0,1  }      & \Vo{0,18 }    & \Vo{0,27 }   \\
         \Se{50 }       & \Vo{1,6 }       & \Vo{3,53}      & \Vo{5,32 }        &  &  & \Se{210}    & \Vo{0,09 }      & \Vo{0,15 }    & \Vo{0,23 }   \\
         \Se{60 }       & \Vo{1,4 }       & \Vo{2,86}      & \Vo{4,32 }        &  &  & \Se{220}    & \Vo{0,073}      & \Vo{0,13 }    & \Vo{0,19 }   \\
         \Se{70 }       & \Vo{1,2 }       & \Vo{2,35}      & \Vo{3,51 }        &  &  & \Se{230}    & \Vo{0,06 }      & \Vo{0,10 }    & \Vo{0,164}   \\
         \Se{80 }       & \Vo{1,0 }       & \Vo{1,89}      & \Vo{2,86 }        &  &  & \Se{240}    & \Vo{0,053}      & \Vo{0,088}    & \Vo{0,138}   \\
         \Se{90 }       & \Vo{0,8 }       & \Vo{1,54}      & \Vo{2,33 }        &  &  & \Se{250}    & \Vo{0,045}      & \Vo{0,077}    & \Vo{0,115}   \\
         \Se{100}       & \Vo{0,7 }       & \Vo{1,27}      & \Vo{1,90 }        &  &  & \Se{260}    & \Vo{0,04 }      & \Vo{0,060}    & \Vo{0,100}   \\
         \Se{110}       & \Vo{0,5 }       & \Vo{1,04}      & \Vo{1,56 }        &  &  & \Se{270}    & \Vo{0,035}      & \Vo{0,055}    & \Vo{0,085}   \\
         \Se{120}       & \Vo{0,4 }       & \Vo{0,84}      & \Vo{1,28 }        &  &  & \Se{280}    & \Vo{0,032}      & \Vo{0,049}    & \Vo{0,073}   \\
         \Se{130}       & \Vo{0,4 }       & \Vo{0,70}      & \Vo{1,05 }        &  &  & \Se{290}    & \Vo{0,027}      & \Vo{0,04 }    & \Vo{0,063}   \\
         \Se{140}       & \Vo{0,3 }       & \Vo{0,57}      & \Vo{0,86 }        &  &  & \Se{300}    & \Vo{0,025}      & \Vo{0,037}    & \Vo{0,055}   \\
         \Se{150}       & \Vo{0,3 }       & \Vo{0,47}      & \Vo{0,71 }        &  &  & \Se{310}    & \Vo{0,023}      & \Vo{0,033}    & \Vo{0,045}   \\ \cline{1-4} \cline{7-10}
 \end{tabular}
 \caption{Messreihen mit Widerstand}

		 \label{tab:messwerte1}
	\end{table}

	\begin{table}
		\centering
\begin{tabular}{@{}ll@{}}
        \toprule
        $t$            & \Vo{10  }     \\ \midrule
        \Se{0}         & \Vo{10,0}     \\
        \Se{30}        & \Vo{9,74}     \\
        \Se{60}        & \Vo{9,46}     \\
        \Se{90}        & \Vo{9,20}     \\
        \Se{120}       & \Vo{8,94}     \\
        \Se{150}       & \Vo{8,70}     \\
        \Se{180}       & \Vo{8,45}     \\
        \Se{210}       & \Vo{8,22}     \\
        \Se{240}       & \Vo{8,00}     \\
        \Se{270}       & \Vo{7,78}     \\
        \Se{300}       & \Vo{7,56}     \\ \bottomrule
\end{tabular}
\caption{Messwerte ohne Widerstand}
\label{tab:messwerte2}

		 \label{tab:messwerte2}
	\end{table}



\newpage
\section{Auswertung}
	\subsection{$I(t)$-Diagramme}
		F�r die $I(t)$-Diagramme werden die Str�me mit der Formel $I=\frac{U}{R}$ berechnet.\\
		Die Str�me zum Zeitpunkt$t$ sind in Tabelle \ref{tab:I(t)} abgebildet.

		\begin{table}
			\centering
\begin{tabular}{llllllllll}
        \cline{1-4} \cline{7-10}
        $t$         & \Vo{5   }     & \Vo{10  }      & \Vo{15   }     &  &  & $t$         & \Vo{5     }     & \Vo{10   }    & \Vo{15   }   \\ \cline{1-4} \cline{7-10}
        \Se{0}      & \Am{5,0 }     & \Am{10,0}      & \Am{15,0 }     &  &  & \Se{160}    & \Am{0,2   }     & \Am{0,38 }    & \Am{0,58 }   \\
        \Se{10}     & \Am{4,0 }     & \Am{8,17}      & \Am{12,16}     &  &  & \Se{170}    & \Am{0,2   }     & \Am{0,31 }    & \Am{0,48 }   \\
        \Se{20}     & \Am{3,2 }     & \Am{6,54}      & \Am{9,85 }     &  &  & \Se{180}    & \Am{0,14  }     & \Am{0,26 }    & \Am{0,40 }   \\
        \Se{30}     & \Am{2,7 }     & \Am{5,30}      & \Am{7,99 }     &  &  & \Se{190}    & \Am{0,12  }     & \Am{0,21 }    & \Am{0,33 }   \\
        \Se{40}     & \Am{2,92}     & \Am{4,34}      & \Am{6,54 }     &  &  & \Se{200}    & \Am{0,1   }     & \Am{0,18 }    & \Am{0,27 }   \\
        \Se{50}     & \Am{1,6 }     & \Am{3,53}      & \Am{5,32 }     &  &  & \Se{210}    & \Am{0,09  }     & \Am{0,15 }    & \Am{0,23 }   \\
        \Se{60}     & \Am{1,4 }     & \Am{2,86}      & \Am{4,32 }     &  &  & \Se{220}    & \Am{0,073 }     & \Am{0,13 }    & \Am{0,19 }   \\
        \Se{70}     & \Am{1,2 }     & \Am{2,35}      & \Am{3,51 }     &  &  & \Se{230}    & \Am{0,06  }     & \Am{0,10 }    & \Am{0,164}   \\
        \Se{80}     & \Am{1,0 }     & \Am{1,89}      & \Am{2,86 }     &  &  & \Se{240}    & \Am{0,053 }     & \Am{0,088}    & \Am{0,138}   \\
        \Se{90}     & \Am{0,8 }     & \Am{1,54}      & \Am{2,33 }     &  &  & \Se{250}    & \Am{0,045 }     & \Am{0,077}    & \Am{0,115}   \\
        \Se{100}    & \Am{0,7 }     & \Am{1,27}      & \Am{1,90 }     &  &  & \Se{260}    & \Am{0,04  }     & \Am{0,060}    & \Am{0,100}   \\
        \Se{110}    & \Am{0,5 }     & \Am{1,04}      & \Am{1,56 }     &  &  & \Se{270}    & \Am{0,035 }     & \Am{0,055}    & \Am{0,085}   \\
        \Se{120}    & \Am{0,4 }     & \Am{0,84}      & \Am{1,28 }     &  &  & \Se{280}    & \Am{0,032 }     & \Am{0,049}    & \Am{0,073}   \\
        \Se{130}    & \Am{0,4 }     & \Am{0,70}      & \Am{1,05 }     &  &  & \Se{290}    & \Am{0,027 }     & \Am{0,04 }    & \Am{0,063}   \\
        \Se{140}    & \Am{0,3 }     & \Am{0,57}      & \Am{0,86 }     &  &  & \Se{300}    & \Am{0,025 }     & \Am{0,037}    & \Am{0,055}   \\
        \Se{150}    & \Am{0,3 }     & \Am{0,47}      & \Am{0,71 }     &  &  & \Se{310}    & \Am{0,023 }     & \Am{0,033}    & \Am{0,045}   \\ \cline{1-4} \cline{7-10}
\end{tabular}
\caption{$I(t)$}
\label{tab:I(t)}

			\label{tab:I(t)}
		\end{table}

		\begin{figure}
		\begin{tikzpicture}[xscale=0.04,yscale=10]
	% Achsen
	\draw[->] (-0.3/0.04,0) -- (320,0) node[below right] {$t$ in \Se{}};
	\draw[->] (0,-0.3/10) -- (0,1.2) node[left] {$I$ in \mAm{}};
	% Achsbeschriftung
	\foreach \x in {50,100,...,310} \draw (\x,0.01) -- (\x,-0.01) node [below] {\x};
	\foreach \y in {0.1,0.2,0.3,0.4,0.5,0.6,0.7,0.8,0.9,1.0,1.1} \draw (3,\y) -- (-3,\y) node [left] {\y};
	% Funktion
	\draw[red, samples=10000, domain=0:320] plot (\x, {0.867324*e^(-0.017749*\x)});
	%Funktionbeschriftung
	\node (32) at (10 , 0.86) [right,red] {$I(t) = 0,867324 \cdot e ^{-0,017749 \cdot t}$};
	% Messpunkte
	\node (1)   at (0    ,1.064) {\textsf{x}};
	\node (2)   at (10  ,0.851) {\textsf{x}};
	\node (3)   at (20  ,0.681) {\textsf{x}};
	\node (4)   at (30  ,0.574) {\textsf{x}};
	\node (5)   at (40  ,0.457) {\textsf{x}};
	\node (6)   at (50  ,0.340) {\textsf{x}};
	\node (7)   at (60  ,0.298) {\textsf{x}};
	\node (8)   at (70  ,0.255) {\textsf{x}};
	\node (9)   at (80  ,0.213) {\textsf{x}};
	\node (10) at (90  ,0.170) {\textsf{x}};
	\node (11) at (100,0.149) {\textsf{x}};
	\node (12) at (110,0.106) {\textsf{x}};
	\node (13) at (120,0.085) {\textsf{x}};
	\node (14) at (130,0.085) {\textsf{x}};
	\node (15) at (140,0.064) {\textsf{x}};
	\node (16) at (150,0.064) {\textsf{x}};
	\node (17) at (160,0.043) {\textsf{x}};
	\node (18) at (170,0.043) {\textsf{x}};
	\node (19) at (180,0.030) {\textsf{x}};
	\node (20) at (190,0.026) {\textsf{x}};
	\node (20) at (200,0.021) {\textsf{x}};
	\node (21) at (210,0.019) {\textsf{x}};
	\node (22) at (220,0.016) {\textsf{x}};
	\node (23) at (230,0.013) {\textsf{x}};
	\node (24) at (240,0.011) {\textsf{x}};
	\node (25) at (250,0.010) {\textsf{x}};
	\node (26) at (260,0.009) {\textsf{x}};
	\node (27) at (270,0.007) {\textsf{x}};
	\node (28) at (280,0.007) {\textsf{x}};
	\node (29) at (290,0.006) {\textsf{x}};
	\node (30) at (300,0.005) {\textsf{x}};
	\node (31) at (310,0.005) {\textsf{x}};
\end{tikzpicture}

		\caption{Diagramm: Kondensatorspannung \Vo{5}}
		\label{fig:Diagramm_5V}
		\end{figure}


		\begin{figure}
		\begin{tikzpicture}[xscale=0.04,yscale=5]
	% Achsen
	\draw[->] (-0.3/0.04,0) -- (320,0) node[below right] {$t$ in \Se{}};
	\draw[->] (0,-0.3/5) -- (0,2.3) node[left] {$I$ in \mAm{}};
	% Achsbeschriftung
	\foreach \x in {50,100,...,310} \draw (\x,0.02) -- (\x,-0.02) node [below] {\x};
	\foreach \y in {0.1,0.2,0.3,0.4,0.5,0.6,0.7,0.8,0.9,1.0,1.1,1.2,1.3,1.4,1.5,1.6,1.7,1.8,1.9,2.0,2.1,2.2} \draw (3,\y) -- (-3,\y) node [left] {\y};
	% Funktion
	\draw[red, samples=10, domain=0:320] plot (\x, {1.842886*e^(-0.018870 * \x)});
	%Funktionbeschriftung
	\node (32) at (10 , 1.85) [right] {$ f(x) = 1.842886 * e ^{ (-0.018870 * x)}$};	
	% Messpunkte
	\node (1)   at (0    ,2.128) {\textsf{x}};
	\node (2)   at (10  ,1.738) {\textsf{x}};
	\node (3)   at (20  ,1.391) {\textsf{x}};
	\node (4)   at (30  ,1.128) {\textsf{x}};
	\node (5)   at (40  ,0.923) {\textsf{x}};
	\node (6)   at (50  ,0.751) {\textsf{x}};
	\node (7)   at (60  ,0.609) {\textsf{x}};
	\node (8)   at (70  ,0.500) {\textsf{x}};
	\node (9)   at (80  ,0.402) {\textsf{x}};
	\node (10) at (90  ,0.328) {\textsf{x}};
	\node (11) at (100,0.270) {\textsf{x}};
	\node (12) at (110,0.221) {\textsf{x}};
	\node (13) at (120,0.179) {\textsf{x}};
	\node (14) at (130,0.149) {\textsf{x}};
	\node (15) at (140,0.121) {\textsf{x}};
	\node (16) at (150,0.100) {\textsf{x}};
	\node (17) at (160,0.081) {\textsf{x}};
	\node (18) at (170,0.066) {\textsf{x}};
	\node (19) at (180,0.055) {\textsf{x}};
	\node (20) at (190,0.045) {\textsf{x}};
	\node (20) at (200,0.038) {\textsf{x}};
	\node (21) at (210,0.032) {\textsf{x}};
	\node (22) at (220,0.028) {\textsf{x}};
	\node (23) at (230,0.021) {\textsf{x}};
	\node (24) at (240,0.019) {\textsf{x}};
	\node (25) at (250,0.016) {\textsf{x}};
	\node (26) at (260,0.013) {\textsf{x}};
	\node (27) at (270,0.012) {\textsf{x}};
	\node (28) at (280,0.010) {\textsf{x}};
	\node (29) at (290,0.009) {\textsf{x}};
	\node (30) at (300,0.008) {\textsf{x}};
	\node (31) at (310,0.007) {\textsf{x}};
\end{tikzpicture}

		\caption{Diagramm: Kondensatorspannung \Vo{10}}
		\label{fig:Diagramm_10V}
		\end{figure}


		\begin{figure}
		\begin{tikzpicture}[xscale=0.04,yscale=5]
	% Achsen
	\draw[->] (-0.3/0.04,0) -- (320,0) node[below right] {$t$ in \Se{}};
	\draw[->] (0,-0.3/5) -- (0,3.3) node[left] {$I$ in \mAm{}};
	% Achsbeschriftung
	\foreach \x in {50,100,...,310} \draw (\x,0.02) -- (\x,-0.02) node [below] {\x};
	\foreach \y in {0.1,0.2,0.3,0.4,0.5,0.6,0.7,0.8,0.9,1.0,1.1,1.2,1.3,1.4,1.5,1.6,1.7,1.8,1.9,2.0,2.1,2.2,2.3,2.4,2.5,2.6,2.7,2.8,2.9,3.0,3.1,3.2} \draw (3,\y) -- (-3,\y) node [left] {\y};
	% Funktion
	\draw[red, samples=10000, domain=0:320] plot (\x, {2.775478*e^(-0.018834*\x)});
	%Funktionbeschriftung
	\node (32) at (10 , 2.75) [right,red] {$I(t) = 2,775478 \cdot e ^{-0,018834 \cdot t}$};
	% Messpunkte
	\node (1)   at (0    ,3.191) {\textsf{x}};
	\node (2)   at (10  ,2.587) {\textsf{x}};
	\node (3)   at (20  ,2.096) {\textsf{x}};
	\node (4)   at (30  ,1.700) {\textsf{x}};
	\node (5)   at (40  ,1.391) {\textsf{x}};
	\node (6)   at (50  ,1.132) {\textsf{x}};
	\node (7)   at (60  ,0.919) {\textsf{x}};
	\node (8)   at (70  ,0.747) {\textsf{x}};
	\node (9)   at (80  ,0.609) {\textsf{x}};
	\node (10) at (90  ,0.496) {\textsf{x}};
	\node (11) at (100,0.404) {\textsf{x}};
	\node (12) at (110,0.332) {\textsf{x}};
	\node (13) at (120,0.272) {\textsf{x}};
	\node (14) at (130,0.223) {\textsf{x}};
	\node (15) at (140,0.183) {\textsf{x}};
	\node (16) at (150,0.151) {\textsf{x}};
	\node (17) at (160,0.123) {\textsf{x}};
	\node (18) at (170,0.102) {\textsf{x}};
	\node (19) at (180,0.085) {\textsf{x}};
	\node (20) at (190,0.070) {\textsf{x}};
	\node (20) at (200,0.057) {\textsf{x}};
	\node (21) at (210,0.049) {\textsf{x}};
	\node (22) at (220,0.040) {\textsf{x}};
	\node (23) at (230,0.035) {\textsf{x}};
	\node (24) at (240,0.029) {\textsf{x}};
	\node (25) at (250,0.024) {\textsf{x}};
	\node (26) at (260,0.021) {\textsf{x}};
	\node (27) at (270,0.018) {\textsf{x}};
	\node (28) at (280,0.016) {\textsf{x}};
	\node (29) at (290,0.013) {\textsf{x}};
	\node (30) at (300,0.012) {\textsf{x}};
	\node (31) at (310,0.010) {\textsf{x}};
\end{tikzpicture}

		\caption{Diagramm: Kondensatorspannung \Vo{15}}
		\label{fig:Diagramm_15V}
		\end{figure}

\newpage
	\subsection{Ladung und Kapazit�t des Kondensators}
		Die Ladung des Kondensators Q ergibt aus der Formel:
		\[Q=\int\limits_{0}^ \infty \mathrm I\,\mathrm dt\]
		Die Kapazit�t des Kondensators C ergibt aus der Formel:
		\[C=\frac{Q}{U}\]



		\subsubsection{F�r Kondensatorspannung 5V}
			Ladung:
			\begin{align}
				Q	& = \int\limits_{0}^ \infty \mathrm 0.867324 \cdot e^{-0.017749 * x}\,\mathrm dt\\
				%Q	& = \SI{\frac{867324}{17749}}{\farad}\\
				Q	& \approx \SI{48.866077}{\coulomb}
			\end{align}

			Kapazit�t:
			\begin{align}
				C	& = \frac{Q}{U}\\
				C	& = \frac{\SI{48.866077}{\coulomb}}{\SI{5}{\volt}}\\
				C	& \approx \SI{9.773215}{\farad}
			\end{align}

		\subsubsection{F�r Kondensatorspannung 10V}
			Ladung:
			\begin{align}
				Q	& = \int\limits_{0}^ \infty \mathrm 1.842886 \cdot e ^{-0.018870 * x}\,\mathrm dt\\
				%Q	& = \SI{\frac{921443}{9435}}{\coulomb}\\
				Q	& \approx \SI{97.662215}{\coulomb}
			\end{align}

			Kapazit�t:
			\begin{align}
				C	& = \frac{Q}{U}\\
				C	& = \frac{\SI{97.662215}{\coulomb}}{\SI{10}{\volt}}\\
				C	& \approx \SI{9.7662212}{\farad}
			\end{align}


		\subsubsection{F�r Kondensatorspannung 15V}
			Ladung:
			\begin{align}
				Q	& = \int\limits_{0}^ \infty \mathrm 2.775478 \cdot e ^{-0.018834 * x}\,\mathrm dt\\
				%Q	& = \SI{\frac{32273}{219}}{\coulomb}\\
				Q	& \approx \SI{147.365297}{\coulomb}
			\end{align}

			Kapazit�t:
			\begin{align}
				C	& = \frac{Q}{U}\\
				C	& = \frac{\SI{147.365297}{\coulomb}}{\SI{15}{\volt}}\\
				C	& \approx \SI{9.824353}{\farad}
			\end{align}



 

\end{document}








