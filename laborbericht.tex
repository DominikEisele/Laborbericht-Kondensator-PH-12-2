\documentclass[a4paper,12pt,fleqn,oneside]{article}
\usepackage{graphicx}
\usepackage{etex}
\usepackage[latin1]{inputenc}
\usepackage[ngerman]{babel}
\usepackage{ae,aecompl}
\usepackage[T1]{fontenc}
\usepackage{ngerman}
\usepackage{float}
\usepackage{ulem}
\usepackage{amssymb}
\usepackage[locale=DE, per-mode=fraction, quotient-mode=fraction, group-minimum-digits=6]{siunitx}
\usepackage{tabularx}
\usepackage{bm}
\usepackage{booktabs}
\usepackage{color}
\usepackage{pictex}
\usepackage[left=2.5cm,right=2.5cm,top=2cm,bottom=2cm,includeheadfoot]{geometry}
\usepackage[section]{placeins}
\usepackage{xspace}
\usepackage{multirow}
\usepackage{lastpage}
\usepackage{fancyhdr}
\usepackage{graphicx}
\usepackage{esvect}
\usepackage{pgfplots}
\usepackage[ngerman]{babel}
\usepackage {graphics}
\usepackage {graphicx}
\usepackage{tikz}
\usepackage{amsmath}
\usepackage{autonum}


\setlength{\headheight}{15pt}
\pagestyle{fancy}
\fancyfoot[C]{Seite \thepage{} von \pageref{LastPage}}
\linespread{1.5}
\author{Dominik Eisele}
\title{Laborbericht}
\date{\today}


%\begingroup\makeatletter\ifx\SetFigFontNFSS\undefined%
%\gdef\SetFigFontNFSS#1#2#3#4#5{%
%  \reset@font\fontsize{#1}{#2pt}%
%  \fontfamily{#3}\fontseries{#4}\fontshape{#5}%
%  \selectfont}%
%\fi\endgroup%


% Zus�tzliche Spalten mit variabler Breite fur tabularx
%\newcolumntype{L}{>{\raggedright\arraybackslash}X} % linksb�ndig
%\newcolumntype{C}{>{\centering\arraybackslash}X} % zentriert
%\newcolumntype{R}{>{\raggedleft\arraybackslash}X} % rechtsb�ndig


\newcolumntype{L}[1]{>{\raggedright\arraybackslash}p{#1}} % linksb�ndig mit Breitenangabe
\newcolumntype{C}[1]{>{\centering\arraybackslash}p{#1}} % zentriert mit Breitenangabe
\newcolumntype{R}[1]{>{\raggedleft\arraybackslash}p{#1}} % rechtsb�ndig mit Breitenangabe


\setlength{\tabcolsep}{0pt}
\renewcommand{\arraystretch}{1}


%\renewcommand*\contentsname{Gliederung}


\let\oldsqrt\sqrt
\def\sqrt{\mathpalette\DHLhksqrt}
\def\DHLhksqrt#1#2{\setbox0=\hbox{$#1\oldsqrt{#2\,}$}\dimen0=\ht0
\advance\dimen0-0.3\ht0
%0.3 ist das Ma� f�r die Hakenl�nge, relativ zum Inhalt der Wurzel
\setbox2=\hbox{\vrule height\ht0 depth -\dimen0}%
{\box0\lower0.4pt\box2}}



\begin{document}

\begin{titlepage}
	\begin{flushleft}
		\vspace*{2\baselineskip}
		{\fontsize{16}{19.2}\selectfont Laborbericht Physik TGE12/2 A}\\[4\baselineskip]
		\begin{tabularx}{\textwidth}{rp{5px}X}
			{\fontsize{16}{19.2}\selectfont Titel:}&&{\fontsize{16}{19.2}\selectfont Entladung eines Kondensators}
		\end{tabularx}
		\\[5\baselineskip]
		\setlength{\tabcolsep}{0pt}
		\renewcommand{\arraystretch}{1,25}
		\begin{tabular}{lp{5px}l}
			{\fontsize{14}{16.8}\selectfont Bearbeiter:}&&{\fontsize{14}{16.8}\selectfont Dominik Eisele{}}\\
			{\fontsize{14}{16.8}\selectfont Mitarbeiter:}&&{\fontsize{14}{16.8}\selectfont }\\
			{\fontsize{14}{16.8}\selectfont Datum Versuchsdurchf�hrung:}&&{\fontsize{14}{16.8}\selectfont 29.02.2016}\\
			{\fontsize{14}{16.8}\selectfont Datum Abgabe:}&&{\fontsize{14}{16.8}\selectfont 23.03.2016}
		\end{tabular}
		\\[2\baselineskip]
		{\fontsize{14}{16.8}\selectfont Ich erkl�re an Eides statt, den vorliegenden Laborbericht selbst angefertigt zu haben. Alle fremden Quellen wurden in diesem Laborbericht benannt.}
		\\[2\baselineskip]
		{\fontsize{14}{16.8}\selectfont Aichwald, 16. M�rz 2016 $  $ Dominik Eisele}
	\end{flushleft}
\end{titlepage}

\setlength{\tabcolsep}{7pt}
\renewcommand{\arraystretch}{1,7}

\newpage
\tableofcontents
\newpage


\section{Einf�hrung}
	Bei dem Versuch zur Entladung eines Kondensators ergibt sich die Entladekurve eines Kondensators �ber einen Widerstand.
	Aus dieser Entladekurve lassen sich nun die Ladung und die Kapazit�t berechnen. \\
	Zus�tzlich wurde f�r den Widerstand, �ber den der Kondensator entladen wird, das Multimetereingesetzt. So kann man den
	Innenwiderstand des Multimeters berechnen.

\subsection{Ziele}
	\begin{itemize}
		\item Aufnahme der Entladekurve eines Kondensators
		\item Bestimmung der Ladung eines Kondensators (aus der Entladekurve)
		\item Bestimmung der Kapazit�t eines Kondensators
		\item Bestimmung des Innenwiderstands eines Multimeters
	\end{itemize}

\subsection{Formeln}
	Strom/Ladung
	\[I=\frac{\Delta Q}{\Delta t} \text{ bzw. }I=\frac{dQ}{dt} \text{; } [I]=\si{\ampere}\]
	Kapazit�t
	\[C=\frac{Q}{U} \text{; } [C]=\si{\farad}\]
	Entladung eines Kondensators �ber einen Widerstand
	\[Q(t)=Q_{0} \cdot e^{-\frac{t}{RC}}\]
	\[U(t)=U_{0} \cdot e^{-\frac{t}{RC}}\]

\newpage
\section{Material und Methoden}

\subsection{Material}
	F�r den Versuch verwendete Materialien:
	\begin{itemize}
		\item 1 $\times$ Netzteil
		\item 1 $\times$ Digitalmultimeter
		\item 1 $\times$ Kondensator
		\item 1 $\times$ Widerstand \SI{47}{\kilo\ohm}
		\item 1 $\times$ Steckbrett
		\item Leitungen
	\end{itemize}

\subsection{Aufbau}
	Die Schaltung, wie sie in Abbildung \ref{fig:skizze_schaltplan} zu sehen ist, wurde auf dem Steckbrett gesteckt. \\
	In der zweiten Messreihe wurde der Widerstand entfernt, sodass der Kondensator nur �ber den Innenwiderstand des
	Multimeters entladen wurde.
	\begin{figure}[H]
		\centering
		\scalebox{1.5}{%Title: /tmp/xfig-fig037018
%%Created by: fig2dev Version 3.2 Patchlevel 5e
%%CreationDate: Thu Mar 31 20:55:23 2016
%%User: dominik@deepthought.private-site.de (Dominik Eisele)
\font\thinlinefont=cmr5
%
\begingroup\makeatletter\ifx\SetFigFont\undefined%
\gdef\SetFigFont#1#2#3#4#5{%
  \reset@font\fontsize{#1}{#2pt}%
  \fontfamily{#3}\fontseries{#4}\fontshape{#5}%
  \selectfont}%
\fi\endgroup%
\mbox{\beginpicture
\setcoordinatesystem units <1.04987cm,1.04987cm>
\unitlength=1.04987cm
\linethickness=1pt
\setplotsymbol ({\makebox(0,0)[l]{\tencirc\symbol{'160}}})
\setshadesymbol ({\thinlinefont .})
\setlinear
%
% Fig POLYLINE object
%
\linethickness= 0.500pt
\setplotsymbol ({\thinlinefont .})
{\color[rgb]{0,0,0}\putrule from  1.611 22.672 to  1.611 22.003
}%
%
% Fig POLYLINE object
%
\linethickness= 0.500pt
\setplotsymbol ({\thinlinefont .})
{\color[rgb]{0,0,0}\putrule from  1.611 23.438 to  1.611 23.973
}%
%
% Fig POLYLINE object
%
\linethickness= 0.500pt
\setplotsymbol ({\thinlinefont .})
{\color[rgb]{0,0,0}\putrule from  0.212 22.441 to  0.212 22.003
\putrule from  0.212 22.003 to  2.974 22.003
\putrule from  2.974 22.003 to  2.974 22.441
}%
%
% Fig POLYLINE object
%
\linethickness= 0.500pt
\setplotsymbol ({\thinlinefont .})
{\color[rgb]{0,0,0}\putrule from  0.212 23.535 to  0.212 23.973
\putrule from  0.212 23.973 to  2.974 23.973
\putrule from  2.974 23.973 to  2.974 23.535
}%
%
% Fig TEXT object
%
\put{R} [lB] at  3.260 22.9
%
% Fig TEXT object
%
\put{C} [lB] at  2.007 22.9
%
% Fig TEXT object
%
\put{V} [lB] at  0.667 22.9
%
% Fig POLYLINE object
%
\linethickness= 0.500pt
\setplotsymbol ({\thinlinefont .})
{\color[rgb]{0,0,0}\putrule from  2.974 23.535 to  2.974 23.317
}%
%
% Fig POLYLINE object
%
\linethickness= 0.500pt
\setplotsymbol ({\thinlinefont .})
{\color[rgb]{0,0,0}\putrule from  1.611 23.110 to  1.611 23.438
}%
%
% Fig ELLIPSE
%
\linethickness= 0.500pt
\setplotsymbol ({\thinlinefont .})
{\color[rgb]{0,0,0}\ellipticalarc axes ratio  0.322:0.328  360 degrees 
	from  0.533 22.989 center at  0.212 22.989
}%
%
% Fig POLYLINE object
%
\linethickness= 0.500pt
\setplotsymbol ({\thinlinefont .})
{\color[rgb]{0,0,0}\putrule from  1.399 23.110 to  1.831 23.110
}%
%
% Fig POLYLINE object
%
\linethickness= 0.500pt
\setplotsymbol ({\thinlinefont .})
{\color[rgb]{0,0,0}\putrule from  1.399 23.000 to  1.831 23.000
}%
%
% Fig POLYLINE object
%
\linethickness= 0.500pt
\setplotsymbol ({\thinlinefont .})
{\color[rgb]{0,0,0}\putrule from  1.611 23.000 to  1.611 22.672
}%
%
% Fig POLYLINE object
%
\linethickness= 0.500pt
\setplotsymbol ({\thinlinefont .})
{\color[rgb]{0,0,0}\putrule from  0.212 22.659 to  0.212 22.441
}%
%
% Fig POLYLINE object
%
\linethickness= 0.500pt
\setplotsymbol ({\thinlinefont .})
{\color[rgb]{0,0,0}\plot  0.212 22.989  0.114 22.879 /
}%
%
% Fig POLYLINE object
%
\linethickness= 0.500pt
\setplotsymbol ({\thinlinefont .})
{\color[rgb]{0,0,0}\putrule from  0.212 23.317 to  0.212 23.535
}%
%
% Fig POLYLINE object
%
\linethickness= 0.500pt
\setplotsymbol ({\thinlinefont .})
{\color[rgb]{0,0,0}\plot  0.212 22.989  0.432 23.207 /
%
% arrow head
%
\plot  0.315 23.015  0.432 23.207  0.238 23.092 /
%
}%
%
% Fig POLYLINE object
%
\linethickness= 0.500pt
\setplotsymbol ({\thinlinefont .})
{\color[rgb]{0,0,0}\putrule from  2.974 22.659 to  2.974 22.441
}%
%
% Fig POLYLINE object
%
\linethickness= 0.500pt
\setplotsymbol ({\thinlinefont .})
{\color[rgb]{0,0,0}\color[rgb]{0,0,0}\putrectangle corners at  2.866 23.317 and  3.084 22.659
}%
\linethickness=0pt
\putrectangle corners at -0.127 23.999 and  3.291 21.977
\endpicture}
}
		\caption{Skizze des Schaltplans}
		\label{fig:skizze_schaltplan}
	\end{figure}


\newpage
\subsection{Durchf�hrung}
	Nachdem die Schaltung wie beschrieben aufegebaut worden war, wurde der Kondensator f�r \SI{30}{\second} mit \SI{5}{\volt}aufgeladen.
	Als der Kondensator vollst�ndig geladen war, wurde das Netzteil ausgesteckt, sodass sich der Kondensator entladen konnte.
	Die Spannung die am Kondensator anlag wurde, w�hrend des Entladevorgangs, in \SI{10}{\second} Intervallen gemessen. Die
	Messungen wurden abgebrochen, nachdem \SI{3}{\percent} der Ausgangsspannung  anlagen, bzw. \SI{5}{\minute} vergangen waren.\\
	Diese Messung wurde mit \SI{10}{\volt} und mit \SI{15}{\volt} wiederholt.\\
	Zus�tzlich wurde eine Messreihe durchgef�hrt, bei der der Widerstand entfernt wurde. Damit hat man die M�glichkeit den
	Innenwiderstand des Multimeters zu berechnen. Diese Messreihe wurde bei eine Audgangsladung von \SI{10}{\volt} durchgef�hrt.


\newpage
\section{Messwerte}
	In Tabelle \ref{tab:messwerte1} sind die Messwerte f�r die ersten drei Messreihen zu sehen. Hierbei handelt es sich um die
	Kondensatorentladungen �ber den Widerstand. In Tabelle \ref{tab:messwerte2} sind die Messwerte f�r die Entladung �ber den
	Innenwiderstand des Multimeters zu sehen.
		 

 
	\begin{table}
		\centering
		\label{tab:messwerte1}
		\begin{tabular}{llllllllll}
			\cline{1-4} \cline{7-10}
			$t$ in $\si{\second}$  	 & $\SI{5}{\volt}$ 	 & $\SI{10}{\volt}$ 	 & $\SI{15}{\volt}$ 	 &  &  & $t$ in $\si{\second}$ & $\SI{5}{\volt}$ & $\SI{10}{\volt}$ & $\SI{15}{\volt}$ \\ \cline{1-4} \cline{7-10}
			\num{0} 	       	       & 5,0 	       	       & 10,0 	       	       & 15,0	       	       &  &  & \num{160} 	       & 0,2	       	       	       & 0,38 	       	       & 0,58 \\
			\num{10}	       	       & 4,0 	       	       & 8,17 	       	       & 12,16 	       	       	       &  &  & \num{170} 	       & 0,2 	       	       & 0,31 	       	       & 0,48 \\
			\num{20}	       	       & 3,2 	       	       & 6,54 	       	       & 9,85 	       	       &  &  & \num{180} 	       & 0,14 	       	       & 0,26 	       	       & 0,40 \\
			\num{30}	       	       & 2,7 	       	       & 5,30 	       	       & 7,99	       	       &  &  & \num{190} 	       & 0,12 	       	       & 0,21 	       	       & 0,33 \\
			\num{40}	       	       & 2,92	       	       & 4,34 	       	       & 6,54	       	       &  &  & \num{200} 	       & 0,1 	       	       & 0,18 	       	       & 0,27 \\
			\num{50}	       	       & 1,6 	       	       & 3,53 	       	       & 5,32	       	       &  &  & \num{210} 	       & 0,09 	       	       & 0,15 	       	       & 0,23 \\
			\num{60}	       	       & 1,4 	       	       & 2,86 	       	       & 4,32	       	       &  &  & \num{220} 	       & 0,073 	       	       	       & 0,13 	       	       & 0,19 \\
			\num{70}	       	       & 1,2 	       	       & 2,35 	       	       & 3,51 	       	       &  &  & \num{230} 	       & 0,06 	       	       & 0,10 	       	       & 0,164\\
			\num{80}	       	       & 1,0 	       	       & 1,89 	       	       & 2,86 	       	       &  &  & \num{240} 	       & 0,053 	       	       	       & 0,088 	       	       	       & 0,138\\
			\num{90}	       	       & 0,8 	       	       & 1,54 	       	       & 2,33 	       	       &  &  & \num{250} 	       & 0,045 	       	       	       & 0,077 	       	       	       & 0,115\\
			\num{100}	       	       & 0,7 	       	       & 1,27 	       	       & 1,90 	       	       &  &  & \num{260} 	       & 0,04 	       	       & 0,060 	       	       	       & 0,100\\
			\num{110}	       	       & 0,5 	       	       & 1,04 	       	       & 1,56 	       	       &  &  & \num{270} 	       & 0,035 	       	       	       & 0,055 	       	       	       & 0,085\\
			\num{120}	       	       & 0,4 	       	       & 0,84 	       	       & 1,28 	       	       &  &  & \num{280} 	       & 0,032 	       	       	       & 0,049 	       	       	       & 0,073\\
			\num{130}	       	       & 0,4 	       	       & 0,70 	       	       & 1,05 	       	       &  &  & \num{290} 	       & 0,027 	       	       	       & 0,04 	       	       & 0,063\\
			\num{140}	       	       & 0,3 	       	       & 0,57 	       	       & 0,86 	       	       &  &  & \num{300}	       & 0,025 	       	       	       & 0,037 	       	       	       & 0,055\\
			\num{150}	       	       & 0,3 	       	       & 0,47 	       	       & 0,71 	       	       &  &  & \num{310}	       & 0,023 	       	       	       & 0,033 	       	       	       & 0,045\\ \cline{1-4} \cline{7-10}
		\end{tabular}
		\caption{Messreihen mit Widerstand}
	\end{table}

	\begin{table}
		\centering
		\label{tab:messwerte2}
		\begin{tabular}{@{}ll@{}}
			\toprule
			$t$ in $\si{\second}$ & $\SI{10}{\volt}$ \\ \midrule
			\num{0}             & 10,0             \\
			\num{30}            & 9,74             \\
			\num{60}            & 9,46             \\
			\num{90}            & 9,20             \\
			\num{120}           & 8,94             \\
			\num{150}           & 8,70             \\
			\num{180}           & 8,45             \\
			\num{210}           & 8,22             \\
			\num{240}           & 8,00             \\
			\num{270}           & 7,78             \\
			\num{300}           & 7,56             \\ \bottomrule
		\end{tabular}
		\caption{Messwerte ohne Widerstand}
	\end{table}

	 

\newpage
\section{Auswertung}
	\subsection{Betrag der elektrischen Feldst�rke $E$}
	Die ben�tigte Formel zu Berechnung der elektrischen Feldst�rke $E$ lautet:
	\[E=\frac{\text{Potentialdifferenz }\Delta\varphi}{\text{Abstand der Potentiallinien }\Delta s}\]
	F�r den Abstand der Potentiallinie  wurde die L�nge Feldlinien gemessen.
 
	\subsubsection{Messreihe 1}
	Punkt $P_1 \left(\num{1.8} \mid 2\right)$,  Punkt $P_2 \left(\num{5,5} \mid \num{4,4}\right)$, Punkt $P_3 \left(\num{9.6} \mid 8\right)$:
		\[E_{P_1} = \frac{\SI{10}{\volt}}{\SI{0,11}{\metre}} = \SI{90,9091}{\volt \per \meter}\]
		\[E_{P_2} = \frac{\SI{10}{\volt}}{\SI{0.115}{\metre}} = \SI{86,9565}{\volt \per \meter}\]
		\[E_{P_3} = \frac{\SI{10}{\volt}}{\SI{0.12}{\metre}} = \SI{83,3333}{\volt \per \meter}\]
 

	\subsubsection{Messreihe 2}
	Punkt $P_1 \left(\num{1,3} \mid \num{5,5}\right)$,  Punkt $P_2 \left(\num{6.2} \mid \num{1,35}\right)$, Punkt $P_3 \left(\num{7.38} \mid \num{-2,05}\right)$:
		\[E_{P_1} = \frac{\SI{10}{\volt}}{\SI{0,12}{\metre}} = \SI{83,3333}{\volt \per \meter}\]
		\[E_{P_2} = \frac{\SI{10}{\volt}}{\SI{0.11}{\metre}} = \SI{90,9091}{\volt \per \meter}\]
		\[E_{P_3} = \frac{\SI{10}{\volt}}{\SI{0.11}{\metre}} = \SI{90,9091}{\volt \per \meter}\]

	\subsection{Beobachtung zum Metallring}
	Der in Messreihe 2 hinzugef�gte Metallring besa� ein einheitliches Potential von  \SI{4,3}{\volt}. Diese einheitliches Potential
	entsteht, da der Ring als Leiter keine Potentialdifferenz besitzt. Dadurch ist die Ringinnenfl�che komplett von den beiden Elektroden abgeschirmt,
	und es wirkt nur der Ring als �quipotentiallinie auf das Ringinnere. Dadurch entsteht keine Potentialdifferenz im Ringinneren.

\end{document}








